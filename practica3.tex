\documentclass[12pt,a4paper]{article}

\usepackage[utf8]{inputenc}
\usepackage{graphicx}
\usepackage[spanish]{babel}
\usepackage{float}				%Para poner las imagenes exactamente donde se me cante las pelotas en caso de quererlo, poniendole [H]
\usepackage{amsmath}
\usepackage{epstopdf}
\usepackage{geometry}
\usepackage{hieroglf}
\usepackage{subcaption}
\usepackage[justification=centering]{caption}
\usepackage[colorlinks=true, allcolors=blue]{hyperref}
\geometry{
a4paper,
left=20mm,
right=20mm,
top=25mm,
bottom = 20mm
}
\usepackage{float}
\usepackage{units}
\marginparwidth=2cm
\usepackage[colorinlistoftodos]{todonotes}

% \usepackage{hyperref}   %Esto es para ir a los links

\title{\mathbf{Mecánica de Medios Continuos \\Práctica 3 \\ Descripción de la deformación}}
\author{Universidad de Cuenca}
\begin{document}
\maketitle
\begin{enumerate}
    \item Se sabe que el tensor de tensiones de Cauchy para cierto medio continuo
    (un fluido) viene dado por $\boldsymbol{\sigma} = -p\mathbf{I} + \lambda Tr(\mathbf{d})\mathbf{I} + 2\mu\mathbf{d}$, donde \textbf{d} es el
    tensor de velocidad de deformación, p es la presión hidrostática y $\lambda$, $\mu$ son
    constantes. Dado un sistema de coordenadas cartesianas xyz, cuya base
    ortonormal correspondiente es ${\mathbf{e_1, e_2, e_3}}$, considere que el campo de
    velocidades viene dado por $\mathbf{v} = kx\mathbf{e_1} + ky\mathbf{e_2} + kz\mathbf{e_3}$. Tome la constante k
    igual al $(n + 2)-1$, donde n es el  ́último dígito de su número de cédula.
    Calcule la expresión matricial del tensor de Cauchy en la base dada.
    \item Considere un medio continuo cuya ecuación del movimiento es
    \begin{equation}
        \mathbf{x}(\mathbf{X},t)=e^{mt}\mathbf{X}
    \end{equation}
    
    donde m es igual al último dígito de su número de cédula más uno.
    \begin{enumerate}
        \item Encuentre la expresión general del estiramiento.
        \item Encuentre el volumen que tendría para $t=5$ un medio continuo que en la
        configuración de referencia es una esfera de radio 1.
    \end{enumerate}
   \item Considere un medio continuo cuya ecuación del movimiento es
   \begin{equation}
       \mathbf{x}(\mathbf{X},t)=(mt+1)\mathbf{X}
   \end{equation}
   
   donde m es igual al último dígito de su número de cédula más uno.
   \begin{enumerate}
       \item Encuentre la expresión del alargamiento unitario para $t=2$.
       \item Encuentre el volumen que tendría para $t=5$ un medio continuo que en la
       configuración de referencia es una esfera de radio 1.
   \end{enumerate}

\end{enumerate}
\end{document}