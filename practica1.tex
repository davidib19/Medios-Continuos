\documentclass[12pt,a4paper]{article}

\usepackage[utf8]{inputenc}
\usepackage{graphicx}
\usepackage[spanish]{babel}
\usepackage{float}				%Para poner las imagenes exactamente donde se me cante las pelotas en caso de quererlo, poniendole [H]
\usepackage{amsmath}
\usepackage{epstopdf}
\usepackage{geometry}
\usepackage{hieroglf}
\usepackage{subcaption}
\usepackage[justification=centering]{caption}
\usepackage[colorlinks=true, allcolors=blue]{hyperref}
\geometry{
a4paper,
left=20mm,
right=20mm,
top=25mm,
bottom = 20mm
}
\usepackage{float}
\usepackage{units}
\marginparwidth=2cm
\usepackage[colorinlistoftodos]{todonotes}

% \usepackage{hyperref}   %Esto es para ir a los links

\title{\mathbf{Mecánica de Medios Continuos \\Práctica 1 \\ Descripción del movimiento}}
\author{Universidad de Cuenca}
\begin{document}
\maketitle
\begin{enumerate}
    \item Dado el movimiento descrito por:
        \begin{equation}
            x_1=e^{t}X_1+X_3,\quad x_2=X_2,\quad x_3=X_3-tX_1.
        \end{equation}
        \begin{enumerate}
            \item Encontrar la descripción espacial del campo B=$X_1+t$.
            \item Encontrar la descripción material del campo c=$x_1+t$.
        \end{enumerate}
    \item Sea el campo bidimensional de velocidades $\mathbf{v}=(v_0,\frac{tx_1}{T})$, donde $v_0$ y $T$ son constantes, calcular las líneas de corriente y especificar la que pasa por el origen de coordenadas en t=0. Calcular lo mismo para las trayectorias.
    \item Para el campo de velocidades en tres dimensiones:
    \begin{equation}
        \mathbf{v}=\left(\frac{x_1}{1+t},\frac{2x_2}{1+t},\frac{3x_3}{1+t}\right),
    \end{equation}
    obtener las componentes de aceleración. Determinar las líneas de corriente y las trayectorias.
    \item Justificar si son ciertas o falsas las siguientes afirmaciones:
    \begin{enumerate}
        \item Si el campo de velocidades es estacionario, el campo de aceleraciones también lo es.
        \item Si el campo de velocidades es uniforme, el campo de aceleraciones es siempre nulo.
        \item Si el campo de velocidades es estacionario y el medio es incompresible, el campo de aceleraciones es siempre nulo.
    \end{enumerate}
    \item Considere que, para un tiempo t dado, la velocidad en descripción material de un medio continuo viene dada por:
        \begin{equation}
            \mathbf{V}=ae^{at}\mathbf{X}.
        \end{equation}
        Donde a es una constante.
        \begin{enumerate}
            \item Encuentre las ecuaciones inversas del movimiento.
            \item Determine si el campo de velocidades es estacionario.
            \item Encuentre la aceleración en descripción espacial.
        \end{enumerate}
        \item Sea un campo de velocidades cuya descripción espacial viene dada por:
            \begin{equation}
                \mathbf{v}=\left(mx_1,nx_2,m+n\right).
            \end{equation}
            donde m y n son constantes.
            \begin{enumerate}
                \item Encuentre las ecuaciones de las líneas de corriente.
                \item Encuentre la aceleración en descripción material.
                \item Encuentre la aceleración en descripción espacial.
            \end{enumerate}
\end{enumerate}
\end{document}