\documentclass[12pt,a4paper]{article}

\usepackage[utf8]{inputenc}
\usepackage{graphicx}
\usepackage[spanish]{babel}
\usepackage{float}				%Para poner las imagenes exactamente donde se me cante las pelotas en caso de quererlo, poniendole [H]
\usepackage{amsmath}
\usepackage{epstopdf}
\usepackage{geometry}
\usepackage{hieroglf}
\usepackage{subcaption}
\usepackage[justification=centering]{caption}
\usepackage[colorlinks=true, allcolors=blue]{hyperref}
\geometry{
a4paper,
left=20mm,
right=20mm,
top=25mm,
bottom = 20mm
}
\usepackage{float}
\usepackage{units}
\marginparwidth=2cm
\usepackage[colorinlistoftodos]{todonotes}

% \usepackage{hyperref}   %Esto es para ir a los links

\title{\textbf{Mécanica de Medios Continuos}}
\author{Universidad de Cuenca}
\begin{document}
\maketitle
\section*{Trayectoria  y Líneas de corriente }
La Trayectoria es el camino que sigue una partícula fija y obedece la ecuación \ref{eq:Trayectoria}.
\begin{equation}
    \frac{d\underline{x}}{dt}=\underline{v}(\underline{x},t) \underline{x}(0)=\underline{X}
    \label{eq:Trayectoria}
\end{equation}
La linea de corriente es una linea tangente a las velocidades que tienen las partículas en un momento fijo y sigue la ecuación \ref{eq:lineacorriente}.
\begin{equation}
    \frac{d\underline{x}}{d\lambda}(\lambda)=\underline{v}(\underline{x}(\lambda);t*) \underline{x}(0)=\underline{x*}
    \label{eq:lineacorriente}
\end{equation}
\textbf{Ejemplo} $v_1=x_1 v_2=x_k$ y $v_3=3x_3^2$.\\
k:constante\\
$[\underline{v}]=\begin{bmatrix}
    x1\\k\\3x_3^2
\end{bmatrix}=\begin{bmatrix}
    v_1\\v_2\\v_3
\end{bmatrix}$\\
Tratectorias:\\
$dx_i/dt=v_i$\\
$x_i(0)=X_i$
\begin{align}
    \frac{dx_1}{dt}&=x_1\\
    \frac{dx_2}{dt}&=k\\
    \frac{dx_3}{dt}&=3x_3^2
\end{align}
\begin{align}
    x_1&=\pm e^{s+t}\\
    x_2&=kt+C_2\\
    x_3&=\frac{-1}{3t+C_3}
\end{align}
\end{document}